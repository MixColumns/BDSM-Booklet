\documentclass[a4paper,12pt]{article}

\usepackage[margin=2.5cm]{geometry}
\usepackage[pdftex]{hyperref}
\usepackage{xcolor}
\usepackage{array}
\usepackage{arydshln}
\usepackage{enumitem}
\usepackage[ngerman]{babel}
\usepackage{hyperref}
\usepackage{datetime}
\usepackage{qrcode}
\usepackage{fancyhdr}

%%%%%%%%%%%%%%%

\begin{document}

\setlength\dashlinedash{0.2pt}
\setlength\dashlinegap{1.5pt}
\setlength\arrayrulewidth{0.3pt}
\newcolumntype{P}[1]{>{\centering\arraybackslash}p{10em}}

%%%%%%%%%%%%%%% Title Settings

\title{BDSM Booklet}
\author{Nicole / CoffeeFox}
\date{\today \ \currenttime \bigbreak \qrcode[hyperlink,height=3cm]{https://github.com/MixColumns/BDSM-Booklet/releases/download/main/main.pdf}}
\renewcommand*\contentsname{Inhaltsverzeichnis}

\pagestyle{fancy}
\fancyhead{} % clear all header fields
\fancyhead[RO,LE]{\textbf{BDSM Booklet}}
\fancyfoot{} % clear all footer fields
\fancyfoot[LE,RO]{\thepage}
\fancyfoot[LO,CE]{ \qrcode[hyperlink,height=1cm]{https://github.com/MixColumns/BDSM-Booklet/releases/download/main/main.pdf}}
%\fancyfoot[CO,RE]{Center Text e.g. Credit}

%%%%%%%%%%%%%%% Make Title

\maketitle
\tableofcontents
\newpage

%%%%%%%%%%%%%%% Allgemeines

\section{Allgemeines}
Diese Sammlung soll dabei helfen das Spielen in der Szene sicherer und einfacher zu machen. Die wichtigsten Themen sollten enthalten sein und die Formulare kännen digital oder manuell ausgefüllt werden und anschließend an Partnerpersonen versendet, in der Cloud geteilt, oder in einem Ordner mitgenommen werden. Prinzipiell ist es nicht nötig alle Formular und Seiten auszufüllen. Es können nach Bedarf einzelne Seiten genutzt werden. Ich habe versucht dies beim Design bestmöglich zu berücksichtigen.\bigbreak

Die Inspiration für dieses Booklet habe ich unter anderem durch LilaHexes\footnote{\url{https://lilahexe.top/}} \glqq Consent Formular\grqq \footnote{\url{https://codeberg.org/LilaHexe/ConsentForm}}, einem Post von Avacyn\footnote{[Indizierte Plattform].com/users/7507485/posts/9366912}, sowie zahlreicher Webformulare erhalten. Mir haben dort jedoch meist Punkte gefehlt und manche Punkte wollte ich gerne anders umsetzen. Es ist also nicht verwunderlich, wenn sich ähnliche Formulierungen und Inhalte auch in ähnlichen Formularen wiederfinden. Zusätzlich sind die Themen Absprache und Konsent auch für mich sehr wichtig und dieses Booklet ist daher auch für mich sehr hilfreich. \bigbreak

Das Booklet wird basierend auf Feedback hoffentlich\textsuperscript{\texttrademark} laufend aktualisiert. Die Version des Dokuments ist durch den Zeitstempel auf dem Deckblatt gekennzeichnet.

\subsection{Technische Details}\label{Technische_Details}
Der \glqq normale\grqq\footnote{Ich meine damit Leser, die nicht an den technischen Details interessiert sind, außer sie haben gewisse Freude daran, sich damit selbst zu foltern.} Leser möge diesen Part überspringen.

Das gesamte Formular ist in \LaTeX\footnote{Nicht das kinky Material, sondern Lamport TeX: \url{https://www.latex-project.org/}} erstellt worden. Das hat gewisse Vorteile, beispielsweise, dass die PDF-Datei\footnote{Ursprünglich ISO 32000, inzwischen gibt es mehrere Standards: \url{https://www.adobe.com/de/acrobat/resources/document-files/pdf-types.html}} automatisch erstellt werden und ich eine Versionskontrolle\footnote{Beispielsweise git: \url{https://git-scm.com/}} für das Dokument benutzen kann. Dadurch können dann auch relativ einfach mehrere Menschen ihre Verbesserungen zeitgleich vorschlagen, auch wenn es Überschneidungen gibt\footnote{Stichwort Mergekonflikte} und Änderungen einfach wieder Rückgängig gemacht werden.

Unter anderem aus Zeitgründen habe ich für die Erstellung mancher Texte und Strukturen auch auf ChatGPT\footnote{\url{https://openai.com/}} zurückgegriffen. Übersetzungen habe ich zum Teil mit DeepL\footnote{\url{https://www.deepl.com/}} durchgeführt.

Das Projekt nutzt die MIT Lizenz\footnote{\url{https://www.tldrlegal.com/license/mit-license}}.

\subsection{Kontakt und Mitarbeit}
Wer den Abschnitt \ref{Technische_Details} nicht übersprungen hat, kann gerne direkt über die Projektseite \url{https://github.com/MixColumns/BDSM-Booklet} mitarbeiten und Anfragen stellen. Ansonsten ist auch eine Kontaktaufnahme über das Profil von CoffeeFox\footnote{[Indizierte Plattform].com/users/17638126} möglich.

%%%%%%%%%%%%%%% Persönliche Angaben

\newpage
\section{Persönliche Angaben}
\subsection{Aktualität}
\begin{tabular}{P|l}
    Formular Zuletzt aktualisiert/ausgefüllt: & \TextField[name=MeLastUpdate,width=25em]{} \\
    \hdashline
\end{tabular}
\subsection{Allgemeines}
\begin{Form}
\begin{tabular}{P|l}
    Name / Szenename & \TextField[name=MeNick,width=25em]{}\\
    \hdashline
    Geburtsdatum / Geburtsjahr & \TextField[name=MeContact,width=25em]{} \\
    \hdashline
    Bevorzugter Name: & \TextField[name=MeBestName,width=25em]{} \\
    \hdashline
    Bevorzugte Pronomen: & \TextField[name=MePronouns,width=25em]{} \\
    \hdashline
    Kontaktmöglichkeiten: & \TextField[name=MeOtherContact,width=25em]{} \\
    \hdashline
    Lieblingstreffpunkte & \TextField[name=MeLocation,multiline=true,height=6em, width=25em]{} \\
    \hdashline
    Weitere Angaben & \TextField[name=MeMore,multiline=true,height=6em, width=25em]{} \\
    \hdashline
\end{tabular}
\end{Form}

\vspace*{2cm}\noindent\textbf{Hier hast du Platz für Fotos, Zeichnungen oder ähnliches:}

%%%%%%%%%%%%%%% Notfall Kontaktpersonen

\newpage
\section{(Notfall) Kontakpersonen}
Kontaktpersonen sind von zentraler Bedeutung, da sie Sicherheit, Unterstützung und emotionale Stabilität bieten. Diese Personen – oft als \glqq Safe Call\grqq oder Vertrauensperson bezeichnet – können in riskanteren Situationen oder auch bei Treffen mit neuen Partner*innen im Notfall eingreifen. Ihre Rolle stärkt das Vertrauen und ermöglicht es Beteiligten, ihre Vorlieben in einem geschützten Rahmen auszuleben. Bei Bedarf können diese Personen kontaktiert werden.
\textbf{Im folgenden Abschnitt kannst du eine primäre und eine sekundäre Kontaktperson auflisten.}\footnote{Das Formular befindet sich auf der nächsten Seite, damit du nach dem Ausdrucken einfacher Seiten aktualisieren kannst. So musst du diesen Text nicht ebenfalls drucken und kannst die Notfallkontakte ohne den Informationstext aufbewahren.}

\newpage
\subsection{Primäre Kontaktperson}
\begin{Form}
	\begin{tabular}{P|l}
		Name & \TextField[name=Contact1Name,width=25em]{}\\
		\hdashline
		Telefonnummer & \TextField[name=Contact1Telefon,width=25em]{} \\
		\hdashline
		Anmerkungen und weitere Kontaktdaten & \TextField[name=Contact1Kommentar,multiline=true,height=6em, width=25em]{} \\
		\hdashline
	\end{tabular}
\end{Form}

\subsubsection{Diese Person sollte kontaktiert werden, wenn...}
\TextField[name=Contact1KontaktWenn,multiline=true,height=7em, width=37em]{}

\subsubsection{Diese Person darf folgendes erfahren:}

\begin{Form}
	\CheckBox[name=Contact1UnfallJ]{Ja}
	\CheckBox[name=Contact1UnfallN]{Nein}
	\noindent Dass ein Unfall passiert ist.
	\newline
	\CheckBox[name=Contact1BDSMJ]{Ja}
	\CheckBox[name=Contact1BdsmN]{Nein}
	\noindent Dass der Unfall wegen einer BDSM Praktik passiert ist.
	\newline
	\CheckBox[name=Contact1BDSMUnfallJ]{Ja}
	\CheckBox[name=Contact1BDSMUnfallN]{Nein}
	\noindent Welche BDSM Praktik den Unfall ausgelöst hat.
	\newline
	\CheckBox[name=Contact1UnfallWerJ]{Ja}
	\CheckBox[name=Contact1UnfallWerN]{Nein}
	\noindent Mit wem der Unfall passiert ist.
	\newline
	\CheckBox[name=Contact1KrankenhausJ]{Ja}
	\CheckBox[name=Contact1KrankenhausN]{Nein}
	\noindent In welchen Krankenhaus man sich befindet (falls relevant)?
\end{Form}

\subsubsection{Weiteres}

\CheckBox[name=Contact1LuegenJ]{Ja}
\CheckBox[name=Contact1LuegenN]{Nein}
\noindent Darf gelogen werden was passiert ist, falls notwendig?
\bigbreak

\noindent \textbf{Weitere Anmerkungen:} \newline
\noindent \TextField[name=Contact1Anmerkungen,multiline=true,height=7em, width=37em]{}

\newpage
\subsection{Sekundäre Kontaktperson}
\begin{Form}
	\begin{tabular}{P|l}
		Name & \TextField[name=Contact2Name,width=25em]{}\\
		\hdashline
		Telefonnummer & \TextField[name=Contact2Telefon,width=25em]{} \\
		\hdashline
		Anmerkungen und weitere Kontaktdaten & \TextField[name=Contact2Kommentar,multiline=true,height=6em, width=25em]{} \\
		\hdashline
	\end{tabular}
\end{Form}

\subsubsection{Diese Person sollte kontaktiert werden, wenn...}
\TextField[name=Contact2KontaktWenn,multiline=true,height=7em, width=37em]{}

\subsubsection{Diese Person darf folgendes erfahren:}

\begin{Form}
	\CheckBox[name=Contact2UnfallJ]{Ja}
	\CheckBox[name=Contact2UnfallN]{Nein}
	\noindent Dass ein Unfall passiert ist.
	\newline
	\CheckBox[name=Contact2BDSMJ]{Ja}
	\CheckBox[name=Contact2BdsmN]{Nein}
	\noindent Dass der Unfall wegen einer BDSM Praktik passiert ist.
	\newline
	\CheckBox[name=Contact2BDSMUnfallJ]{Ja}
	\CheckBox[name=Contact2BDSMUnfallN]{Nein}
	\noindent Welche BDSM Praktik den Unfall ausgelöst hat.
	\newline
	\CheckBox[name=Contact2UnfallWerJ]{Ja}
	\CheckBox[name=Contact2UnfallWerN]{Nein}
	\noindent Mit wem der Unfall passiert ist.
	\newline
	\CheckBox[name=Contact2KrankenhausJ]{Ja}
	\CheckBox[name=Contact2KrankenhausN]{Nein}
	\noindent In welchen Krankenhaus man sich befindet (falls relevant)?
\end{Form}

\subsubsection{Weiteres}

\CheckBox[name=Contact2LuegenJ]{Ja}
\CheckBox[name=Contact2LuegenN]{Nein}
\noindent Darf gelogen werden was passiert ist, falls notwendig?
\bigbreak

\noindent \textbf{Weitere Anmerkungen:} \newline
\noindent \TextField[name=Contact2Anmerkungen,multiline=true,height=7em, width=37em]{}

%%%%%%%%%%%%%%% Gesundheitszustand

\newpage
\section{Gesundheitszustand}
Angaben zum aktuelle Gesundheitszustand sind wichtig, da sie sicherstellen, dass alle Beteiligten mögliche Risiken kennen und minimieren können. Informationen zu physischen und psychischen Gesundheitsproblemen, Medikamenten oder Allergien helfen dabei, Aktivitäten anzupassen, um Verletzungen oder negative Reaktionen zu vermeiden. Das Formular ermöglicht es, die Grenzen und Bedürfnisse aller sicher und verantwortungsvoll zu berücksichtigen, was essenziell für das Wohlbefinden und die Sicherheit ist.\textbf{Im folgenden Abschnitt kannst du also deinen Gesundheitszustand auflisten.}\footnote{Das Formular befindet sich auf der nächsten Seite, damit du nach dem Ausdrucken einfacher Seiten aktualisieren kannst. So musst du diesen Text nicht ebenfalls drucken und kannst die Informationen zum Gesundheitszustand ohne den Informationstext aufbewahren.}

\newpage
\subsection{Hast du aktuell akute gesundheitliche Beschwerden?}
\begin{Form}
\begin{tabular}{P|l}
    Nein & \CheckBox[name=GesundheitBeschwerdenN]{}\\
    Ja, und zwar: & \TextField[name=GesundheitBeschwerdenJ,multiline=true,height=6em, width=25em]{} \\
    Im Notfall ist folgendes zu tun: & \TextField[name=GesundheitBeschwerdenNotfall,multiline=true,height=6em, width=25em]{} \\
\end{tabular}
\end{Form}

\subsection{Hast du chronische gesundheitliche Beschwerden?}
Inbesondere Sachen die einen medizinischen Notstand auslösen können. Dazu zählen beispielsweise Asthma, COPD, Diabetes, Herzschwäche, Epilepsie, oder ähnliches. \\
\begin{Form}
\begin{tabular}{P|l}
    Nein & \CheckBox[name=GesundheitChronischN]{}\\
    Ja, und zwar: & \TextField[name=GesundheitChronisch,multiline=true,height=6em, width=25em]{} \\
    Im Notfall ist folgendes zu tun: & \TextField[name=GesundheitChronischNotfall,multiline=true,height=6em, width=25em]{} \\
\end{tabular}
\end{Form}

\subsection{Hast du aktuell Verletzungen oder körperliche Einschränkungen?}
\begin{Form}
\begin{tabular}{P|l}
    Nein & \CheckBox[name=GesundheitVerletztN]{}\\
    Ja, und zwar: & \TextField[name=GesundheitVerletzt,multiline=true,height=6em, width=25em]{} \\
    Diese schränken mich wie folgt ein: & \TextField[name=GesundheitVerletztEinschr,multiline=true,height=6em, width=25em]{} \\
\end{tabular}
\end{Form}

\newpage
\subsection{Hast du Allergien?}
\begin{Form}
\begin{tabular}{P|l}
    Nein & \CheckBox[name=GesundheitAllergieN]{}\\
    Ja, und zwar: & \TextField[name=GesundheitAllergie,multiline=true,height=6em, width=25em]{} \\
    Im Notfall ist folgendes zu tun: & \TextField[name=GesundheitAllergieNotfall,multiline=true,height=6em, width=25em]{} \\
\end{tabular}
\end{Form}

\subsection{Benötigst du regelmäßige Medikamente?}
\begin{Form}
\begin{tabular}{P|l}
    Nein & \CheckBox[name=GesundheitMedikamenteRegelN]{}\\
    Ja, diese befinden sich ... und ich benötige sie ... oft: & \TextField[name=GesundheitMedikamenteRegel,multiline=true,height=6em, width=25em]{} \\
\end{tabular}
\end{Form}

\subsection{Benötigst du Notfallmedikamente?}
\begin{Form}
\begin{tabular}{P|l}
    Nein & \CheckBox[name=GesundheitMedikamenteNotfallN1n]{}\\
    Ja, diese befinden sich ... und ich benötige sie wenn ... : & \TextField[name=GesundheitMedikamenteNotfall,multiline=true,height=6em, width=25em]{} \\
\end{tabular}
\end{Form}

\subsection{Hast du besondere Trigger oder Traumata?}
\begin{Form}
\begin{tabular}{P|l}
    Nein & \CheckBox[name=GesundheitTriggerN]{}\\
    Ja, und zwar: & \TextField[name=GesundheitTrigger,multiline=true,height=6em, width=25em]{} \\
\end{tabular}
\end{Form}

\newpage

\subsection{Hast du Kreislaufprobleme?}
\begin{Form}
\begin{tabular}{P|l}
    Nein & \CheckBox[name=GesundheitKreislaufN]{}\\
    Ja & \CheckBox[name=GesundheitKreislaufJ]{}\\
\end{tabular}
\end{Form}

\subsection{Hast du eine Blutgerinnungsstörung?}
\begin{Form}
\begin{tabular}{P|l}
    Nein & \CheckBox[name=GesundheitBlutN]{}\\
    Ja & \CheckBox[name=GesundheitBlutJ]{}\\
\end{tabular}
\end{Form}

\subsection{Hast du Erste-Hilfe-Wissen?}
\begin{Form}
\begin{tabular}{P|l}
    Nein & \CheckBox[name=GesundheitErsteHilfeN]{}\\
    Ja & \CheckBox[name=GesundheitErsteHilfeJ]{}\\
\end{tabular}
\end{Form}

\subsection{Wurdest du schon einmal auf STIs getestet?}
\textbf{Mögliche Tests sind beispielsweise}: HIV, Hepatitis, Syphilis, Gonorrhoe, Chlamydien und Sonstige.\bigbreak
\begin{Form}
\begin{tabular}{P|l}
    Nein & \CheckBox[name=GesundheitSTITestN]{}\\
    Ja, am ... auf ... mit folgendem Ergebnis: & \TextField[name=GesundheitSTITest,multiline=true,height=6em, width=25em]{} 
    \\
\end{tabular}
\end{Form}

\subsection{Weitere Anmerkungen}
\noindent \TextField[name=GesundheitAnmerkungen,multiline=true,height=17em, width=37em]{}

%%%%%%%%%%%%%%% Aftercare

\newpage
\section{Aftercare}
Aftercare spielt eine entscheidende Rolle, um sicherzustellen, dass alle Beteiligten nach einer Szene emotional und physisch unterstützt werden. Es fördert das Wohlbefinden, stärkt die Beziehung zwischen den Beteiligten und reduziert potenzielle negative Folgen einer intensiven Erfahrung. Dennoch ist aftercare immer individuell und es ist nicht immer möglich im Vorhinein zu sagen, wie aftercare aussehen sollte. Daher bieten die folgenden Punkte einen Leitfaden an dem sich ein Partner orientieren kann.
\textbf{Im folgenden Abschnitt kannst du also deine Aftercare-Präferenzen auflisten.}\footnote{Das Formular befindet sich auf der nächsten Seite, damit du nach dem Ausdrucken einfacher Seiten aktualisieren kannst. So musst du diesen Text nicht ebenfalls drucken und kannst die Aftercareliste ohne den Informationstext aufbewahren.}
\newpage

\subsection{Emotionale Bedürfnisse}
\noindent Gibt es bestimmte emotionale Reaktionen, auf die man vorbereitet sein sollte (z. B. Traurigkeit, Euphorie, Erschöpfung)? Wie kann man nach der Szene emotional unterstützen? Beispiele: Nähe, Lob, ruhige Gespräche, in Ruhe gelassen werden, Rückversicherungen vom Gegenüber, ... \newline
\noindent \TextField[name=AftercareEmotional,multiline=true,height=7em, width=37em]{}

\subsection{Physische Bedürfnisse}
\noindent Hierzu zählen beispielsweise Getränke oder Snacks (z. B. Wasser, Tee, Schokolade, Traubenzucker), Wärme (z. B. Decke, Wärmflasche), Hygiene (z. B. Duschen, Wundversorgung) und medizinische Maßnahmen oder erste Hilfe, die möglicherweise erforderlich ist. \newline
\noindent \TextField[name=AftercarePhysisch,multiline=true,height=7em, width=37em]{}

\subsection{Zeitlicher Rahmen}
\noindent Wie lange benötigst du in der Regel Aftercare? Sofort nach der Szene oder nach einem längeren Zeitraum? Möchtest du in den kommenden Tagen kontaktiert werden, z. B. per Nachricht oder Telefon, um über die Szene zu sprechen oder zu reflektieren? Möchtest du vielleicht lieber zunächst in Ruhe gelassen werden? \newline
\noindent \TextField[name=AftercareZeitlich,multiline=true,height=7em, width=37em]{}

\subsection{Emotionaler Rückzug oder Isolation}
\noindent Gibt es eine Tendenz zu emotionalem „Drop“? Wenn ja, wie kann das aufgefangen werden? Welche Anzeichen deuten darauf hin, dass du Unterstützung benötigst? \newline
\noindent \TextField[name=AftercareDrop,multiline=true,height=7em, width=37em]{}

\subsection{Persönliche Vorlieben}
\noindent Was hilft dir, dich zu entspannen oder zu erholen? Beispiele: Musik hören, lesen, ein Bad nehmen. Gibt es Dinge, die dir helfen, dich sicher und wohl zu fühlen (z. B. ein Kuscheltier, Lieblingsdecke)? \newline
\noindent \TextField[name=AftercarePreference,multiline=true,height=7em, width=37em]{}

\subsection{Kommunikation}
\noindent Gibt es Wörter, Phrasen oder Namen, die beruhigend wirken? Möchtest du ein ausführliches Gespräch über die Szene führen? Wenn ja, wann und in welcher Form? Ist deine Kommunikation immer verbal? Falls nicht, welche Hilfen gibt es für den Partner?\newline
\noindent \TextField[name=AftercareKommunikation,multiline=true,height=7em, width=37em]{}

\subsection{Besondere Hinweise}
\noindent Gibt es Themen oder Handlungen, die nach einer Szene nicht angesprochen oder ausgeführt werden sollten? Kann es passieren, dass nach einer Szene professionelle Unterstützung notwendig wird? Wer wäre hier Ansprechpartner? Gibt es generelle Anmerkungen?\newline
\noindent \TextField[name=AftercareBesonders,multiline=true,height=7em, width=37em]{}

\end{document}
